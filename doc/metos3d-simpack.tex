%
% Metos3D: A Marine Ecosystem Toolkit for Optimization and Simulation in 3-D
% Copyright (C) 2012  Jaroslaw Piwonski, CAU, jpi@informatik.uni-kiel.de
%
% This program is free software: you can redistribute it and/or modify
% it under the terms of the GNU General Public License as published by
% the Free Software Foundation, either version 3 of the License, or
% (at your option) any later version.
%
% This program is distributed in the hope that it will be useful,
% but WITHOUT ANY WARRANTY; without even the implied warranty of
% MERCHANTABILITY or FITNESS FOR A PARTICULAR PURPOSE.  See the
% GNU General Public License for more details.
%
% You should have received a copy of the GNU General Public License
% along with this program.  If not, see <http://www.gnu.org/licenses/>.
%

%
%	document class
%
\documentclass{article}
%
%	packages
%
\usepackage{amsmath}
\usepackage[pdfborder={0 0 0},colorlinks,urlcolor=blue]{hyperref}
%
%	commands
%
\parindent0em
%
%	BEGIN DOCUMENT
%
\begin{document}

%
%	title
%
\title{
Metos3D\\
\medskip
\texttt{simpack}
}
\author{
Jaroslaw Piwonski\thanks{\texttt{jpi@informatik.uni-kiel.de}} \,,
Thomas Slawig\thanks{\texttt{ts@informatik.uni-kiel.de},
both: Department of Computer Science, Algorithmic Optimal Control -- Computational Marine Science,
Excellence Cluster The Future Ocean, Christian-Albrechts-Platz 4, 24118 Kiel, Germany.}
}
\date{\today}
\maketitle

%
%	Introduction		%%%%%%%%%%%%%%%%%%%%%%%%%%%%%%%%%%%%%%%%%%%%
%
\newpage
\section{Introduction}

The Simulation Package of Metos3D is a software for computation and simulation of
steady periodic cycles of passive biogeochemical tracers. It complies to a general
programming interface, embedding the biogeochemical modelling into an
optimization context.
Metos3D is founded on the PETSc library \cite{PETSc} as a basis for efficient,
parallelized numerical methods. Moreover it's portable and relies only on software,
which is freely and public available.

The application of marine transport is based on the so-called Transport Matrix Method.
The idea and its use are presented in \cite{KhViCa05}, \cite{Kha07} and \cite{Kha08}.

%
%	Quick start	%%%%%%%%%%%%%%%%%%%%%%%%%%%%%%%%%%%%%%%%%%%%%%%%
%
\section{Quick start}

\begin{enumerate}
%
\item
Install \href{http://www.mcs.anl.gov/petsc/}{PETSc}, 
\href{http://ftp.mcs.anl.gov/pub/petsc/release-snapshots/petsc-3.3-p7.tar.gz}{version 3.3}.
%
\item
Visit the Metos3D web page \href{https://metos3d.github.com/metos3d/}{https://metos3d.github.com/metos3d/}
and follow the instructions to install Metos3D.
\end{enumerate}

Now, assuming the PETSc variables (\texttt{PETSC\_DIR} and \texttt{PETSC\_ARCH}) are set and
\texttt{metos3d} is installed as a shell command. 
%
To compile the driver routines with the \texttt{MITgcm-PO4-DOP} model and
run the executable with the provided test option file just type:
%
\begin{verbatim}
$>
metos3d simpack MITgcm-PO4-DOP
$>
./metos3d-simpack-MITgcm-PO4-dop.exe \
model/MITgcm-PO4-DOP/option/test.MITgcm-PO4-DOP.option.txt
\end{verbatim}

%
%	Formats	%%%%%%%%%%%%%%%%%%%%%%%%%%%%%%%%%%%%%%%%%%%%%%%%%%%%
%
\newpage
\section{Formats}

Metos3D is scalable to a great extent. You can choose a free number of tracers,
boundary and/or domain conditions for example. Mostly you will use files to
provide and/or store data. And mostly your data file names will be
enumerated. In that case Metos3D gives you the possibility to provide \emph{formats}.
Note that
an enumeration in Metos3D is always \textbf{zero indexed},
i.e. for a given count $n$ it starts at $0$ and ends at $n-1$.

\bigskip

\textbf{Format:} \\
A Metos3D format is a character string, that \textbf{optionally} embeds an \emph{integer format tag}.
Note that the formatting of a format string \textbf{without} a tag is the format string
itself. This is particularly desired for the case, where $ n = 1 $.

\bigskip

\textbf{Integer Format Tag:} \\
A Metos3D integer format tag starts with a '\texttt{\$}', ends with a '\texttt{d}'
and \textbf{optionally} consist of a width specification in between.

\bigskip

\textbf{Examples:} \\
\\
\begin{tabular}{r|r|r|r|r|r|r}
			& \multicolumn{6}{l}{Format}																												\\ \cline{2-7}
			& \texttt{\$d}			& \texttt{\$02d}		& \texttt{tr\$02d.petsc}	& \texttt{\$003d}	& \texttt{\$0004d}	& \texttt{sp\$0004d.petsc}	\\ \hline
Enumeration	& \multicolumn{6}{l}{Formatting}																											\\ \hline
0			& \texttt{0}				& \texttt{00}		& \texttt{tr00.petsc}	& \texttt{000}		& \texttt{0000}		& \texttt{sp0000.petsc}		\\
$\vdots$		& $\vdots$				& $\vdots$			& $\vdots$				& $\vdots$			& $\vdots$			& $\vdots$					\\
9			& \texttt{9}				& \texttt{09}		& \texttt{tr09.petsc}	& \texttt{009}		& \texttt{0009}		& \texttt{sp0009.petsc}		\\
10			& \texttt{10}			& \texttt{10}		& \texttt{tr10.petsc}	& \texttt{010}		& \texttt{0010}		& \texttt{sp0010.petsc}		\\
$\vdots$		& $\vdots$				& $\vdots$			& $\vdots$				& $\vdots$			& $\vdots$			& $\vdots$					\\
99			& \texttt{99}			& \texttt{99}		& \texttt{tr99.petsc}	& \texttt{099}		& \texttt{0099}		& \texttt{sp0099.petsc}		\\
100			& \texttt{100}			& \texttt{100}		& \texttt{tr100.petsc}	& \texttt{100}		& \texttt{0100}		& \texttt{sp0100.petsc}		\\
$\vdots$		& $\vdots$				& $\vdots$			& $\vdots$				& $\vdots$			& $\vdots$			& $\vdots$					\\
999			& \texttt{999}			& \texttt{999}		& \texttt{tr999.petsc}	& \texttt{999}		& \texttt{0999}		& \texttt{sp0999.petsc}		\\
1000			& \texttt{1000}			& \texttt{1000}		& \texttt{tr1000.petsc}	& \texttt{1000}		& \texttt{1000}		& \texttt{sp1000.petsc}		\\
$\vdots$		& $\vdots$				& $\vdots$			& $\vdots$				& $\vdots$			& $\vdots$			& $\vdots$					
\end{tabular}

%
%	Options	%%%%%%%%%%%%%%%%%%%%%%%%%%%%%%%%%%%%%%%%%%%%%%%%%%%%
%
\newpage
\section{Options}

Once you have successfully compiled Metos3D you can use an \emph{option file}
to control the software. The option file consists of the following options.
\textbf{Comma separated lists must not contain spaces.} 

%
%	Debug
%
\subsection{Debug}

The debugging option specifies the amount of information Metos3D gives you
about the \emph{program flow}. You can find more monitoring options in Section \ref{Solver}. \\

\emph{Option name:}
\begin{verbatim}
-Metos3DDebugLevel
\end{verbatim}

\emph{Option value:} \\
\vspace{-0.3cm}\\
- a positive integer (or zero), specifying the amount of output \\
\\
\begin{tabular}{c|l}
Value & Description \\ \hline
\texttt{0} & no output \\
\texttt{1} & output at main routine entry points \\
\texttt{2} & additional output at subroutine entry points \\
\texttt{3} & additional output during time stepping
\end{tabular} \\

%
%	Geometry
%
\subsection{Geometry}

The geometry options specify the type of geometry,
the input directory and the names of the index files. \\

% type
\emph{Option name:}
\begin{verbatim}
-Metos3DGeometryType                                
\end{verbatim}

\emph{Option value:} \\
\vspace{-0.3cm}\\
- a character string, specifying the geometry type \\
\\
\begin{tabular}{c|l}
Value & Description \\ \hline
\texttt{Profile} & geometry is described by profiles (see \cite{KhViCa05})
\end{tabular} \\

% data
\emph{Option name:}
\begin{verbatim}
-Metos3DProfileInputDirectory
-Metos3DProfileIndexStartFile
-Metos3DProfileIndexEndFile
\end{verbatim}

\emph{Option value:} \\
\vspace{-0.3cm}\\
- a character string, specifying the input directory as an absolute or a relative path \\
- a character string, specifying the name of the start index file \\
- a character string, specifying the name of the end index file

%
%	Tracer
%
\subsection{Tracer}

% count
The tracer options specify the number of tracers, the initial values or files and
the input and output directories. \\

\emph{Option name:}
\begin{verbatim}
-Metos3DTracerCount
\end{verbatim}

\emph{Option value:} \\
\vspace{-0.3cm}\\
- a positive integer, specifying the number of tracers \\

% init files
If you decide to provide the tracer initial values by files,
use the following options. Metos3D will search for the input directory option
first. If present, it searches for the file format option first and then
for the file name list option. \\

\emph{Option name:}
\begin{verbatim}
-Metos3DTracerInputDirectory
-Metos3DTracerInitFileFormat
-Metos3DTracerInitFile                              
\end{verbatim}

\emph{Option value:} \\
\vspace{-0.3cm}\\
- a character string, specifying the input directory as an absolute or a relative path \\
- a character string, specifying the format of the initial value files \\
- a comma separated list of character strings, specifying the names of the initial value files \\

% init values
Alternatively you can omit the options above and provide a list of initial values only.
Metos3D will initialize the tracers with the corresponding (constant) concentration. \\

\emph{Option name:}
\begin{verbatim}
-Metos3DTracerInitValue
\end{verbatim}

\emph{Option value:} \\
\vspace{-0.3cm}\\
- a comma separated list of real values, specifying the initial (constant) tracer concentrations \\

% output files
Moreover you can provide options specifying the output directory and the output files.
Again, Metos3D searches for the output directory option first. If present,
it searches for the file format option first and then for the file name list. \\
%If not present, no data will be written to disk. \\

\emph{Option name:}
\begin{verbatim}
-Metos3DTracerOutputDirectory
-Metos3DTracerOutputFileFormat
-Metos3DTracerOutputFile
\end{verbatim}

\emph{Option value:} \\
\vspace{-0.3cm}\\
- a character string, specifying the output directory as an absolute or a relative path \\
- a character string, specifying the format of the output files \\
- a comma separated list of character strings, specifying the names of the output files \\

%
%	Parameter
%
\subsection{Parameter}

The parameter options specify the number of parameters for the BGC model
and their values. \\

% count
\emph{Option name:}
\begin{verbatim}
-Metos3DParameterCount
\end{verbatim}

\emph{Option value:} \\
\vspace{-0.3cm}\\
- a positive integer (or zero), specifying the number of parameters \\

% values
\emph{Option name:}
\begin{verbatim}
-Metos3DParameterValue
\end{verbatim}

\emph{Option value:} \\
\vspace{-0.3cm}\\
- a list of comma separated real values, specifying the parameter values \\

%
%	Boundary Conditions
%
\subsection{Boundary Conditions}

The boundary condition options specify the number of distinct boundary conditions,
their input directory, their names, the number of data files per condition
and the file format per condition. \\

% count
\emph{Option name:}
\begin{verbatim}
-Metos3DBoundaryConditionCount
\end{verbatim}

\emph{Option value:} \\
\vspace{-0.3cm}\\
- a positive integer (or zero), specifying the number of boundary conditions \\

If you don't need any boundary condition, provide zero as count.
In this case you can skip the following options. \\

% input, names
\emph{Option name:}
\begin{verbatim}
-Metos3DBoundaryConditionInputDirectory
-Metos3DBoundaryConditionName
\end{verbatim}

\emph{Option value:} \\
\vspace{-0.3cm}\\
- a character string, specifying the input directory as an absolute or a relative path \\
- a comma separated list of character strings, specifying the names of the boundary conditions \\

The following options have to be provided for each boundary condition
that was named by the boundary condition name list above. \\

% name, file format
\emph{Option name:}
\begin{verbatim}
-Metos3D...Count
-Metos3D...FileFormat
\end{verbatim}

\emph{Option value:} \\
\vspace{-0.3cm}\\
- a positive integer, specifying the number of data files for the corresponding boundary condition \\
- a character string, specifying the format of the data files for the corresponding boundary condition \\

\textbf{Example:}
\begin{quote}
Assume one of your boundary condition is named '\texttt{IceCover}'. Then the options
above will be '\texttt{-Metos3DIceCoverCount}' and '\texttt{-Metos3DIceCoverFileFormat}'.
\end{quote}

%
%	Domain Conditions
%
\subsection{Domain Conditions}

The domain condition options specify the number of distinct domain conditions,
their input directory, their names, the number of data files per condition
and the file format per condition. \\

% count
\emph{Option name:}
\begin{verbatim}
-Metos3DDomainConditionCount
\end{verbatim}

\emph{Option value:} \\
\vspace{-0.3cm}\\
- a positive integer (or zero), specifying the number of domain conditions \\

If you don't need any domain condition, provide zero as count.
In this case you can skip the following options. \\

% input, names
\emph{Option names:}
\begin{verbatim}
-Metos3DDomainConditionInputDirectory
-Metos3DDomainConditionName
\end{verbatim}

\emph{Option value:} \\
\vspace{-0.3cm}\\
- a character string, specifying the input directory as an absolute or a relative path \\
- a comma separated list of character strings, specifying the names of the domain conditions \\

The following options have to be provided for each domain condition
that was named by the domain condition name list above. \\

% name, file format
\emph{Option names:}
\begin{verbatim}
-Metos3D...Count
-Metos3D...FileFormat
\end{verbatim}

\emph{Option value:} \\
\vspace{-0.3cm}\\
- a positive integer, specifying the number of data files for the corresponding domain condition \\
- a character string, specifying the format of the data files for the corresponding domain condition \\

\textbf{Example:}
\begin{quote}
Assume one of your domain condition is named '\texttt{Heights}'. Then the options
above will be '\texttt{-Metos3DHeightsCount}' and '\texttt{-Metos3DHeightsFileFormat}'.
\end{quote}

%
%	Transport
%
\subsection{Transport}

The transport options specify the transport type,
the input directory of the matrix files, their names and count. \\

% type
\emph{Option name:}
\begin{verbatim}
-Metos3DTransportType                                
\end{verbatim}

\emph{Option value:} \\
\vspace{-0.3cm}\\
- a character string, specifying the transport type \\
\\
\begin{tabular}{c|l}
Value & Description \\ \hline
\texttt{Matrix} & transport is described by matrices (see \cite{KhViCa05})
\end{tabular} \\

% input, count, names
\emph{Option name:}
\begin{verbatim}
-Metos3DMatrixInputDirectory
-Metos3DMatrixCount
-Metos3DMatrixExplicitFileFormat
-Metos3DMatrixImplicitFileFormat
\end{verbatim}

\emph{Option value:} \\
\vspace{-0.3cm}\\
- a character string, specifying the input directory as an absolute or a relative path \\
- a positive integer, specifying the number of matrix files \\
- a character string, specifying the format of the explicit matrix files \\
- a character string, specifying the format of the implicit matrix files \\

%
%	Time Stepping
%
\subsection{Time Stepping}

The time stepping options specify the starting time, the number of
time steps per period and the time step. \\

% start, count, step
\emph{Option name:}
\begin{verbatim}
-Metos3DTimeStepStart
-Metos3DTimeStepCount
-Metos3DTimeStep
\end{verbatim}

\emph{Option value:} \\
\vspace{-0.3cm}\\
- a positive real value within the $[0, 1]$ time interval, specifying the starting time \\
- a positive integer (or zero), specifying the number of time steps per period \\
- a positive real value, specifying the time step \\

%
%	Solver
%
\subsection{Solver}
\label{Solver}

The solver options specifies the solver type used for simulation
(computation of the solution) and the solver type dependent settings. \\

% type
\emph{Option name:}
\begin{verbatim}
-Metos3DSolverType
\end{verbatim}

\emph{Option value:} \\
\vspace{-0.3cm}\\
- a character string, specifying the solver type \\
\\
\begin{tabular}{c|l}
Value & Description \\ \hline
\texttt{Spinup} & fixed point iteration \\
\texttt{Newton} & Newton-Krylov solver from PETSc
\end{tabular} \\

% spin-up 
\subsubsection{Spin-up}

The spin-up options specify the maximum number of periods
and the minimum tolerance for the difference between the
tracer concentrations before and after one period.
If you provide both options, the program stops if \textbf{one} of the conditions
is satisfied. \\

% count, tolerance
\emph{Option name:}
\begin{verbatim}
-Metos3DSpinupCount
-Metos3DSpinupTolerance
\end{verbatim}

\emph{Option value:} \\
\vspace{-0.3cm}\\
- a positive integer (or zero), specifying the number of simulation periods \\
- a positive real value, specifying the tolerance \\

Moreover the spin-up solver gives you the opportunity to monitor the
norm of the residual and to write the whole or parts of the trajectory to disk. \\

% norm
\emph{Option name:}
\begin{verbatim}
-Metos3DSpinupMonitor
\end{verbatim}

\emph{Option value:} \\
\vspace{-0.3cm}\\
- \textbf{no value}, boolean option: If present, the value is \texttt{TRUE} or else \texttt{FALSE}. \\

% prefix
\emph{Option name:}
\begin{verbatim}
-Metos3DSpinupMonitorFileFormatPrefix
\end{verbatim}

\emph{Option value:} \\
\vspace{-0.3cm}\\
- a comma separated \textbf{pair} of character strings, specifying 
the trajectory file name prefix according to current period and
the current time step \\

\textbf{Example:}
\begin{quote}
Assume you have two tracers and you have provided the output file names,
the number of time steps and a spin-up period count with
\begin{verbatim}
-Metos3DTracerOutputFile                 I.petsc,Cs.petsc
-Metos3DTimeStepCount                    2880
-Metos3DSpinupCount                      1
\end{verbatim}
already. The \textbf{additional} option line
\begin{verbatim}
-Metos3DSpinupMonitorFileFormatPrefix    sp$0004d-,ts$0004d-
\end{verbatim}
will instruct Metos3D to create the following trajectory data files
within the tracer output directory:
\begin{verbatim}
sp0000-ts0000-I.petsc ... sp0000-ts2879-I.petsc
sp0000-ts0000-Cs.petsc ... sp0000-ts2879-Cs.petsc
\end{verbatim}
\end{quote}

% modulo step
\emph{Option name:}
\begin{verbatim}
-Metos3DSpinupMonitorModuloStep
\end{verbatim}

\emph{Option value:} \\
\vspace{-0.3cm}\\
- a comma separated \textbf{pair} of positive integers, specifying 
the number of periods and time steps between output \\

\textbf{Example 1:}
\begin{quote}
Assume we have the same situation as above. The \textbf{additional} option line
\begin{verbatim}
-Metos3DSpinupMonitorModuloStep          1,240
\end{verbatim}
will instruct Metos3D to create trajectory data files for every
period, but only every 240 time steps within a period.
\end{quote}

\textbf{Example 2:}
\begin{quote}
Assume we have the same situation as above. The \textbf{additional} option line
\begin{verbatim}
-Metos3DSpinupMonitorModuloStep          12,120
\end{verbatim}
will instruct Metos3D to create trajectory data files for every 12th
period and every 120 time steps within a period.
\end{quote}

% newton
\subsubsection{Newton}

The Newton-Krylov solver options are PETSc options with the
'\texttt{Metos3DNewton\_}' prefix. See the PETSc manual for details.
You can find a short description of the commonly used options
in the following. SNES is the short cut for Scalable Nonlinear
Equation Solver and KSP for Krylov subSPace method. \\

\textbf{SNES options:}

\begin{verbatim}
-Metos3DNewton_snes_type                            ls
\end{verbatim}
The original option name is '\texttt{-snes\_type}'. It specifies the
globalization technique used by the Newton solver. '\texttt{ls}'
means Line Search with cubic backtracking.

\begin{verbatim}
-Metos3DNewton_snes_ksp_ew
\end{verbatim}
The original option name is '\texttt{-snes\_ksp\_ew}'. It specifies,
if the solver algorithm should use the Eisenstatt-Walker \cite{EisWal96}
tolerance control or not.

\begin{verbatim}
-Metos3DNewton_snes_max_it                          1
-Metos3DNewton_snes_max_funcs                       1
\end{verbatim}
The original option names are '\texttt{-snes\_max\_it}' and '\texttt{-snes\_max\_funcs}'.
They specify the maximum number of Newton steps and function evaluations,
respectively.

\begin{verbatim}
-Metos3DNewton_snes_rtol                            1.e-5
-Metos3DNewton_snes_atol                            1.e-3
\end{verbatim}
The original option names are '\texttt{-snes\_rtol}' and '\texttt{-snes\_atol}'.
They specify the minimum relative and absolute tolerance,
respectively.

\begin{verbatim}
-Metos3DNewton_snes_monitor
\end{verbatim}
The original option name is '\texttt{-snes\_monitor}'. It specifies,
if the solver should print out the residual norm every Newton step or not.

\begin{verbatim}
-Metos3DNewton_snes_view
\end{verbatim}
The original option name is '\texttt{-snes\_view}'. It specifies,
if the solver should print out all solver parameters or not. \\

\textbf{KSP options:}

\begin{verbatim}
-Metos3DNewton_ksp_type                             gmres
\end{verbatim}
The original option name is '\texttt{-ksp\_type}'. It specifies the
algorithm for solving the system of linear equations during a
Newton step. '\texttt{gmres}' means Generalized Minimal RESidual method.

\begin{verbatim}
-Metos3DNewton_ksp_max_it                           200
\end{verbatim}
The original option name is '\texttt{-ksp\_max\_it}'. 
It specifies the maximum number of iterations (applications of the
Jacobian to a vector) of the Krylov subspace algorithm.

\begin{verbatim}
-Metos3DNewton_ksp_gmres_restart                    200
\end{verbatim}
The original option name is '\texttt{-ksp\_gmres\_restart}'.
It specifies the maximum number of basis functions for the Krylov subspace.

\begin{verbatim}
-Metos3DNewton_ksp_gmres_modifiedgramschmidt
\end{verbatim}
The original option name is '\texttt{-ksp\_gmres\_modifiedgramschmidt}'.
It specifies, if the solver should use the modified (stabilized) orthogonalization or not.

\begin{verbatim}
-Metos3DNewton_ksp_monitor
\end{verbatim}
The original option name is '\texttt{-ksp\_monitor}'.
It specifies,
if the solver should print out the residual norm every Krylov step or not.

%
%	References	%%%%%%%%%%%%%%%%%%%%%%%%%%%%%%%%%%%%%%%%%%%%%%%%
%
\bibliographystyle{plain}
%\bibliography{/Users/jpicau/Documents/ARBEIT/CODE/Literature/literature}
\begin{thebibliography}{1}

\bibitem{PETSc}
Satish Balay, Kris Buschelman, William~D. Gropp, Dinesh Kaushik, Matthew~G.
  Knepley, Lois~Curfman McInnes, Barry~F. Smith, and Hong Zhang.
\newblock {PETSc Web page}, 2012.
\newblock {http://www.mcs.anl.gov/petsc/ } (last access: 12 July 2013).

\bibitem{EisWal96}
Stanley~C. Eisenstat and Homer~F. Walker.
\newblock Choosing the forcing terms in an inexact newton method.
\newblock {\em SIAM Journal on Scientific Computing}, 17(1):16--32, 1996.

\bibitem{Kha07}
S.~Khatiwala.
\newblock A computational framework for simulation of biogeochemical tracers in
  the ocean.
\newblock {\em Global Biogeochemical Cycles}, 21, 2007.

\bibitem{KhViCa05}
S.~Khatiwala, M.~Visbeck, and M.A. Cane.
\newblock Accelerated simulation of passive tracers in ocean circulation
  models.
\newblock {\em Ocean Modelling}, 9(1):51--69, 2005.

\bibitem{Kha08}
Samar Khatiwala.
\newblock Fast spin up of ocean biogeochemical models using matrix-free
  newton-krylov.
\newblock {\em Ocean Modelling}, 23(3-4):121--129, 2008.

\end{thebibliography}

\end{document}



%
% DUMP
%

%
%\item
%Assuming the PETSc variables (\texttt{PETSC\_DIR} and \texttt{PETSC\_ARCH}) are set and
%the Metos3D installation directory is added to the shell search path variable (\texttt{PATH}),
%i.e. \texttt{metos3d} is installed as shell command:

%Assuming \href{http://www.mcs.anl.gov/petsc/}{PETSc},
%\href{http://ftp.mcs.anl.gov/pub/petsc/release-snapshots/petsc-3.3-p7.tar.gz}{version 3.3},
%is installed do the following:

%\begin{enumerate}
%%
%\item Prepare the \textbf{data}, i.e. download an archive from \\
%		\href{https://github.com/metos3d/data}{https://github.com/metos3d/data},
%		extract it and follow the instructions.
%%
%\item Prepare the \textbf{model}, i.e. download an archive from \\
%		\href{https://github.com/metos3d/model}{https://github.com/metos3d/model}
%		and extract it.
%%
%\item Prepare the \textbf{simulation package}, i.e. download an archive from \\
%		\href{https://github.com/metos3d/simpack}{https://github.com/metos3d/simpack},
%		extract it and change into the package directory.
%%
%\item Link the data and model directories. For example:
%\begin{verbatim}
%$>
%ln -s ../metos3d-data-v0.2 data
%ln -s ../metos3d-model-v0.2 model
%\end{verbatim}
%%
%\item Set the PETSc environment variables. For example:
%\begin{verbatim}
%$>
%. petsc/local.jserver.petsc.opt.txt
%\end{verbatim}
%%
%\item Compile the software with a chosen model. For example:
%\begin{verbatim}
%$>
%make BGC=model/I-Cs/
%\end{verbatim}
%%
%\item Run executable with chosen options. For example:
%\begin{verbatim}
%$>
%./metos3d-simpack-I-Cs.exe model/I-Cs/option/local.jserver.option.I-Cs.simpack.txt
%\end{verbatim}
%%
%\end{enumerate}

